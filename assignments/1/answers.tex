% Options for packages loaded elsewhere
\PassOptionsToPackage{unicode}{hyperref}
\PassOptionsToPackage{hyphens}{url}
%
\documentclass[
]{article}
\usepackage{amsmath,amssymb}
\usepackage{lmodern}
\usepackage{iftex}
\ifPDFTeX
  \usepackage[T1]{fontenc}
  \usepackage[utf8]{inputenc}
  \usepackage{textcomp} % provide euro and other symbols
\else % if luatex or xetex
  \usepackage{unicode-math}
  \defaultfontfeatures{Scale=MatchLowercase}
  \defaultfontfeatures[\rmfamily]{Ligatures=TeX,Scale=1}
\fi
% Use upquote if available, for straight quotes in verbatim environments
\IfFileExists{upquote.sty}{\usepackage{upquote}}{}
\IfFileExists{microtype.sty}{% use microtype if available
  \usepackage[]{microtype}
  \UseMicrotypeSet[protrusion]{basicmath} % disable protrusion for tt fonts
}{}
\makeatletter
\@ifundefined{KOMAClassName}{% if non-KOMA class
  \IfFileExists{parskip.sty}{%
    \usepackage{parskip}
  }{% else
    \setlength{\parindent}{0pt}
    \setlength{\parskip}{6pt plus 2pt minus 1pt}}
}{% if KOMA class
  \KOMAoptions{parskip=half}}
\makeatother
\usepackage{xcolor}
\setlength{\emergencystretch}{3em} % prevent overfull lines
\providecommand{\tightlist}{%
  \setlength{\itemsep}{0pt}\setlength{\parskip}{0pt}}
\setcounter{secnumdepth}{-\maxdimen} % remove section numbering
\ifLuaTeX
  \usepackage{selnolig}  % disable illegal ligatures
\fi
\IfFileExists{bookmark.sty}{\usepackage{bookmark}}{\usepackage{hyperref}}
\IfFileExists{xurl.sty}{\usepackage{xurl}}{} % add URL line breaks if available
\urlstyle{same} % disable monospaced font for URLs
\hypersetup{
  hidelinks,
  pdfcreator={LaTeX via pandoc}}

\author{}
\date{}

\begin{document}

\(\gdef\unit#1{\,\text{#1}}\)

\hypertarget{lecture-1-problems}{%
\section{Lecture 1 Problems}\label{lecture-1-problems}}

\begin{enumerate}
\def\labelenumi{\arabic{enumi}.}
\item
  The equation \[ d_{end-to-end} = N \frac{L}{R} \] Where \(L\) is the
  length of each packet, \(N\) is the number of links, and \(R\) is the
  transmission speed of each link can be generalized for \(P\) packets
  as \[ d_{end-to-end} = (N + P - 1) \frac{L}{R} \] Here,
  \(\frac{L}{R}\) is the transmission delay of each link. Links
  transport 1 packet at a time, and there are \(N - 1\) switches between
  each link.
\item
  ~

  \begin{enumerate}
  \def\labelenumii{\alph{enumii}.}
  \tightlist
  \item
    The network can support 16 simultaneous connections: 4 connections
    between each of the 4 hosts.
  \item
    There can be 8 simultaneous connections between host A and C: 4
    routing through B and another 4 routing through D.
  \item
    Yes, we can route 4 connections between A and C as well as B and D:
  \end{enumerate}
\item
  ~

  \begin{enumerate}
  \def\labelenumii{\alph{enumii}.}
  \tightlist
  \item
    \(d_{prop} = \frac{m}{s}\)
  \item
    \(d_{trans} = \frac{L}{R}\)
  \item
    \(d_{end-to-end} = d_{prop} + d_{trans}\)
  \item
    \(x_{last} = (t-d_{trans})s\), so at time \(t = d_{trans}\),
    \(x_{last} = 0\). In other words, the last bit of the packet is
    still just about to leave \(A\).
  \item
    \(x_{first} = ts\). At time \(t = d_{trans}\),
    \(x_{first} = d_{trans}s\). If \(d_{prop} > d_{trans}\), then it
    follows that \(x_{first} < m\) (because \(s = \frac{m}{d_{prop}}\),
    so \(x_{first} = d_{trans} \frac{m}{d_{prop}}\)), meaning the first
    bit of the packet has yet to reach \(B\).
  \item
    If instead \(d_{prop} < d_{trans}\), then it follows that
    \(x_{first} > m\), meaning the first bit of the packet has reached
    \(B\). \(\
    \gdef\s{2.5 \cdot 10^8 \unit{m/s}}\
    \gdef\L{120 \unit{bits}}\
    \gdef\R{56 \unit{kbps}}\
    \)
  \item
    If \(s = \s\), \(L = \L\), \(R = \R\), and \(d_{prop} = d_{trans}\)
    then \[
    \begin{split}
    \frac{m}{s} &= \frac{L}{R}\\
    &\Downarrow\\
    m &= s \frac{L}{R}\\
    &= (\s) \frac{\L}{\R}\\
    &\approx 535 \unit{m}\\
    \end{split}
    \]
  \end{enumerate}
\item
  If \(N = \text{number of packets}\),
  \(L = \text{length of each packet in bits}\), and
  \(R = \text{transmission rate in bps}\), then

  \begin{enumerate}
  \def\labelenumii{\alph{enumii}.}
  \tightlist
  \item
    \[
    \begin{split}
    d_{avg\,queue} &= \frac{\sum_{i=0}^{N - 1} i \frac{L}{R}}{N}\\
    &= \frac{(N - 1)L}{2R}
    \end{split}
    \]
  \item
    If \(N\) packets arrive to the link every \(\frac{LN}{R}\) seconds,
    then the average queuing delay is still \(\frac{(N - 1)L}{2R}\),
    since the next burst of packets arrives right as the previous burst
    is done.
  \end{enumerate}
\item
  \(\gdef\m{20\,000 \unit{km}}
  \gdef\R{2 \unit{Mbps}}
  \gdef\s{2.5 \cdot 10^8 \unit{m/s}}
  \gdef\bxd{160 \unit{kb}}\) If \(m = \m\), \(R = \R\), and \(s = \s\),
  then

  \begin{enumerate}
  \def\labelenumii{\alph{enumii}.}
  \item
    The bandwidth-delay product \[
    \begin{split}
    R \cdot d_{prop} &= R \frac{m}{s}\\
    &= \R \cdot \frac{\m}{\s}\\
    &= \bxd
    \end{split}
    \]
  \item
    The maximum number of bits in the link at any given must be \(\bxd\)
    (the number of bits the link can transmit in \(d_{prop}\) seconds).
  \item
    The \textbf{bandwidth-delay product} must be the maximum number of
    bits that can fit in a given link at any given time.
  \item
    \(w = \frac{m}{\bxd} = \frac{\m}{\bxd} = 125 \unit{meters per bit}\).
    This would make each bit longer than a football field.
  \item
    \(w = \frac{m}{R \frac{m}{s}} = \frac{s}{R}\)
  \end{enumerate}
\item
  \(\gdef\m{35\,786 \unit{km}}
  \gdef\R{10 \unit{Mbps}}
  \gdef\s{2.4 \cdot 10^8 \unit{m/s}}
  \gdef\D{0.15 \unit{s}}
  \gdef\bxd{150 \unit{Mb}}\) If \(R = \R\), \(s = \s\), and
  \(m = \text{distance from geostationary satellite to surface} = \m\),
  then

  \begin{enumerate}
  \def\labelenumii{\alph{enumii}.}
  \item
    \(d_{prop} = \frac{m}{s} = \frac{\m}{\s} \approx \D\)
  \item
    The bandwidth-delay product
    \(R \cdot d_{prop} = \R \cdot \D = \bxd\).
  \item
    \[
    \begin{split}
    d_{end-to-end} &= d_{trans} + d_{prop}\\
    &= \frac{x}{R} + d_{prop}\\
    x &= R(d_{end-to-end} - d_{prop})\\
    &= \R \cdot (60 \unit{s} - \D)\\
    &= 450 \unit{Mb}
    \end{split}
    \]
  \end{enumerate}
\end{enumerate}

\end{document}
